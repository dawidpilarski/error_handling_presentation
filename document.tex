\documentclass[10pt]{beamer}

\usetheme[progressbar=frametitle, block=fill]{metropolis}
\usepackage{xcolor}
\usepackage{multirow}
\usepackage{pgfpages}
\usepackage{pifont}
\newcommand{\cmark}{\ding{51}}
\newcommand{\xmark}{\ding{55}}
\setbeamertemplate{note page}{\insertnote}
%\setbeameroption{show notes on second screen=left}
\setbeameroption{hide notes}
\definecolor{amethyst}{rgb}{0.5, 0.4, 1.0}
\definecolor{amethystgrey}{rgb}{0.85, 0.85, 1.0}
\definecolor{amethystdark}{rgb}{0.4, 0.3, 0.9}
\definecolor{orangedark}{rgb}{0.0, 0.9, 0}
%\definecolor{titlebg}{HTML}{4e8074}
%\definecolor{titlebg}{HTML}{3e7985}
\definecolor{titlebg}{HTML}{fbf8ff}
\definecolor{font}{HTML}{23373b}

%\setbeamercolor{title}{fg=amethyst, bg=amethyst}
\setbeamercolor{frametitle}{fg= font, bg=titlebg}
%\setbeamercolor{section title}{black}
%\setbeamercolor{structure}{fg=amethyst, bg=amethyst}
\setbeamercolor{progress bar}{ fg = amethyst, bg= amethystgrey }
%\setbeamercolor{itemize item}{fg=amethyst,bg=white}
\setbeamercolor{alerted text}{fg=amethystdark}
%\setbeamercolor{title separator}{ ... }
%\setbeamercolor{progress bar in head/foot}{ ... }
%\setbeamercolor{progress bar in section page}{ ... }

\usepackage[utf8]{inputenc}
\usepackage[english]{babel}

\usepackage{booktabs}
\usepackage[scale=2]{ccicons}

\usepackage{minted}

\usepackage{pgfplots}

\usepackage{xspace}

\title{Futuristic Error Handling}
\subtitle{Error handling in C++ today and tomorrow}
%\logo{\includegraphics[width=0.05\linewidth]{logo.png}}
% \date{\today}
\date{}
\author{Dawid Pilarski}
\institute{dawid.pilarski@panicsofware.com}

\begin{document}

\maketitle

%\begin{frame}{Table of contents}
%  \setbeamertemplate{section in toc}[sections numbered]
%  \tableofcontents[hideallsubsections]
%\end{frame}

\begin{frame}{Why am I here?}
	Why should we bother with error handling?
\end{frame}

\begin{frame}{Recommendable error handling mechanism}
	\note{Q: ask which mechanism would be chosen}
	Which error mechanism would you choose?
	
	There exist two common strategies for error handling:
	\begin{itemize}
		\item error codes?
		\item exceptions?
	\end{itemize}
\end{frame}

\begin{frame}{Who am I?}
	\begin{columns}[onlytextwidth]
		\begin{column}{0.7\textwidth}
			\begin{itemize}
				\item Senior Software Developer in TomTom
				\item Member of the ISO/JTC1/SC22/WG21
				\item C++ blog writer and C++ evangelist
			\end{itemize}
		\end{column}
		\begin{column}{0.29\textwidth}
			\includegraphics[width=\linewidth]{Dawid_Pilarski.jpg}
		\end{column}	
	\end{columns}
\end{frame}

\section{Error codes nowadays}

\begin{frame}{What are the error codes?}
	According to the \alert{Wikipedia} :
	\begin{itemize}[<+- | alert@+>]
		\item error code is an enumerated message,
		\item that corresponds to the status of a specific software application.
		\item They are typically used to identify faults, such as those in faulty hardware, software, or incorrect user input
	\end{itemize}
\end{frame}

\begin{frame}{The error codes.}
	\begin{itemize}[<+- | alert@+>]
		\item Old. C-compatible. Comes from assembly time.
		\item Machine friendly.
		\item Super fast.
		\item Used till today.
	\end{itemize}
\end{frame}

\begin{frame}[fragile]{Error code example}
	\begin{minted}{c++}
int sqlite3_open( const char *filename, sqlite3 **ppDb );
	\end{minted}
	
	\pause
	
	\hrulefill
	
	\begin{minted}[highlightlines={2,4}]{c++}
int open_status = sqlite3_open(/* ... */ );
if(open_status == SQLITE_OK){
  // make use of opened database
} else if( open_status == SQLITE_CANTOPEN_ISDIR ) {
  // handle the error
}
	\end{minted}
	
\end{frame}

\begin{frame}[fragile]{Error code example}
\begin{minted}{c++}
int sqlite3_open( const char *filename, sqlite3 **ppDb );
\end{minted}

\hrulefill

\begin{minted}[highlightlines={3}]{c++}
int open_status = sqlite3_open(/* ... */ );
if(open_status == SQLITE_OK){
  // make use of opened database
} else if( open_status == SQLITE_CANTOPEN_ISDIR ) {
  // handle the error
}
\end{minted}

\end{frame}

\begin{frame}[fragile]{Error code example}
\begin{minted}{c++}
int sqlite3_open( const char *filename, sqlite3 **ppDb );
\end{minted}

\hrulefill

\begin{minted}[highlightlines={5}]{c++}
int open_status = sqlite3_open(/* ... */ );
if(open_status == SQLITE_OK){
// make use of opened database
} else if( open_status == SQLITE_CANTOPEN_ISDIR ) {
// handle the error
}
\end{minted}

\end{frame}


\begin{frame}{Handle the error}
	How to handle the error correctly?
	
	\pause
	
	\begin{itemize}[<+- | alert@+>]
		\item \texttt{std::terminate()}
		\item take the error callback
		\item propagate the error to the caller
	\end{itemize}
	
\end{frame}

\section{Error codes - propagation}

\begin{frame}[fragile]{Propagation}
	\begin{minted}[highlightlines=1]{c++}
void foo_bar(int& errc /*...*/){
  errc = foo();
  // ...
  errc = bar();		
  // ...
}
	\end{minted}
	
\end{frame}

\begin{frame}[fragile]{Propagation}
\begin{minted}[highlightlines={2,4}]{c++}
void foo_bar(int& errc /*...*/){
errc = foo();
// ...
errc = bar();		
// ...
}
\end{minted}

\end{frame}

\begin{frame}[fragile]{Error translation}
	\begin{minted}[highlightlines={3,6}]{c++}
void foo_bar(foo_bar_errc errc&){
  foo_errc ferrc = foo();
  errc = translate_foo(ferrc);
  // ...
  bar_errc berrc = bar();
  errc = translate_foo(berrc);
}
	\end{minted}
\end{frame}

\begin{frame}{C-style error codes summary}
	So we can see {\color{red}serious disadvantages} (except for {\color{blue}obvious advantages}):
	
	\begin{itemize}[<+- | alert@+>]
		\item success path same as error path
		\item boiler plate code
		\item cluttering code with translations
	\end{itemize}
\end{frame}
	

\section{Error codes - modern approach}
\begin{frame}{Standard library support - what do we need?}
	\begin{itemize}
		\item A way to define new error codes
		\item A way to distinguish domain of the error codes
	\end{itemize}
\end{frame}

\begin{frame}{Standard library support - what we get?}
	We get three new major types:
	\begin{itemize}[<+- | alert@+>]
		\item std::error\_code
		\item std::error\_category
		\item std::error\_condition
	\end{itemize}
\end{frame}
	

\begin{frame}[fragile]{std::error\_code in action}
	\begin{minted}{c++}
std::error_code errcode;
is_regular_file("non_existent_directory", errcode);

std::cout << errcode << std::endl;
std::cout << errcode.value() << std::endl;
std::cout << errcode.message() << std::endl;
std::cout << errcode.category().name() << std::endl;
	\end{minted}
	
	\hrulefill
	
	\begin{block}{output}
	\texttt{\\
		\$ generic:2 \\
		\$ 2 \\
		\$ No such file or directory \\
		\$ generic}	
	\end{block}
	
\end{frame}

\begin{frame}[fragile]{Acting upon error}
	\begin{minted}{c++}
std::error_code errcode;
is_regular_file("non_existent_file", errcode);
  
if(errcode == errc::no_such_file_or_directory){
  // creating a file
}
	\end{minted}
\end{frame}

\begin{frame}{Let's define our own error code}
	Steps to create own error code:
	\begin{itemize}[<+- | alert@+>]
		\item define custom enum with error codes
		\item inform, that the enum is an error code
		\item create custom error category (or use existing one)
		\item create enum to error code factory function
		\item define custom error condition
		\begin{itemize}
			\item define error condition enum
			\item inform the world about new error condition enum
			\item make conversion function from new error code to error condition
		\end{itemize}
		\item enjoy!
	\end{itemize}
\end{frame}

\section{Error codes - \\ defining custom error codes}

\begin{frame}[fragile]{Step 1 - define custom enum with error codes}
	\begin{minted}[highlightlines={1}]{c++}
enum class open_file_error {
  SUCCESS, // zero means success
  NO_SUCH_FILE_OR_DIRECTORY,
  FILE_IS_DIRECTORY,
  LACK_OF_RESOURCES,
  FILE_BROKEN,
  NO_PERMISSIONS
};
	\end{minted}
\end{frame}

\begin{frame}[fragile]{Step 1 - define custom enum with error codes}
\begin{minted}[highlightlines={3,4}]{c++}
enum class open_file_error {
  SUCCESS, // zero means success
  NO_SUCH_FILE_OR_DIRECTORY,
  FILE_IS_DIRECTORY,
  LACK_OF_RESOURCES,
  FILE_BROKEN,
  NO_PERMISSIONS
};
\end{minted}
\end{frame}

\begin{frame}[fragile]{Step 1 - define custom enum with error codes}
\begin{minted}[highlightlines={5,6}]{c++}
enum class open_file_error {
  SUCCESS, // zero means success
  NO_SUCH_FILE_OR_DIRECTORY,
  FILE_IS_DIRECTORY,
  LACK_OF_RESOURCES,
  FILE_BROKEN,
  NO_PERMISSIONS
};
\end{minted}
\end{frame}

\begin{frame}[fragile]{Step 1 - define custom enum with error codes}
\begin{minted}[highlightlines={7}]{c++}
enum class open_file_error {
  SUCCESS, // zero means success
  NO_SUCH_FILE_OR_DIRECTORY,
  FILE_IS_DIRECTORY,
  LACK_OF_RESOURCES,
  FILE_BROKEN,
  NO_PERMISSIONS
};
\end{minted}
\end{frame}

\begin{frame}[fragile]{Step 2 - inform the world about new error code type}
\begin{minted}{c++}
namespace std{
  template <> struct
  is_error_code_enum<open_file_error> : std::true_type{};
}
\end{minted}
	
\end{frame}

\begin{frame}[fragile]{Step 3 - custom error category}
	
	\begin{minted}{c++}
struct open_file_error_domain : std::error_category {
  const char *name() const noexcept override;		
  std::string message(int errc) const override;
};
	\end{minted}
\end{frame}

\begin{frame}[fragile]{Step 3 - custom error category}
\begin{minted}{c++}
const char* open_file_error_domain::name() const noexcept{
  return "Open File Error";
}
\end{minted}
\end{frame}

\begin{frame}[fragile]{Step 3 - custom error category}
	\begin{minted}[highlightlines={4,5}]{c++}
std::string open_file_error_domain::message(int errc) const{
  if(errc < 0 or errc > 4) return "UNKNOWN ERROR";
  switch (static_cast<map_access_error>(errc)){
    case open_file_error::SUCCESS:
      return "Success.";
    case open_file_error::NO_SUCH_FILE_OR_DIRECTORY:
      return "File does not exist.";
    case open_file_error::FILE_IS_DIRECTORY:
      return "Cannot open directory.";
    case open_file_error::LACK_OF_RESOURCES:
      return "No available file descriptor";
    case open_file_error::FILE_BROKEN:
      return "File is corrupted."
    case open_file_error::NO_PERMISSIONS:
      return "Missing permissions to open the file."
  }
}
	\end{minted}
\end{frame}

\begin{frame}[fragile]{Step 3 - custom error category}
\begin{minted}[highlightlines={6,7}]{c++}
std::string open_file_error_domain::message(int errc) const{
  if(errc < 0 or errc > 4) return "UNKNOWN ERROR";
  switch (static_cast<map_access_error>(errc)){
    case open_file_error::SUCCESS:
      return "Success.";
    case open_file_error::NO_SUCH_FILE_OR_DIRECTORY:
      return "File does not exist.";
    case open_file_error::FILE_IS_DIRECTORY:
      return "Cannot open directory.";
    case open_file_error::LACK_OF_RESOURCES:
      return "No available file descriptor";
    case open_file_error::FILE_BROKEN:
      return "File is corrupted."
    case open_file_error::NO_PERMISSIONS:
      return "Missing permissions to open the file."
  }
}
\end{minted}
\end{frame}

\begin{frame}[fragile]{Step 3 - custom error category}
\begin{minted}[highlightlines={8-15}]{c++}
std::string open_file_error_domain::message(int errc) const{
  if(errc < 0 or errc > 4) return "UNKNOWN ERROR";
  switch (static_cast<map_access_error>(errc)){
    case open_file_error::SUCCESS:
      return "Success.";
    case open_file_error::NO_SUCH_FILE_OR_DIRECTORY:
      return "File does not exist.";
    case open_file_error::FILE_IS_DIRECTORY:
      return "Cannot open directory.";
    case open_file_error::LACK_OF_RESOURCES:
      return "No available file descriptor";
    case open_file_error::FILE_BROKEN:
      return "File is corrupted."
    case open_file_error::NO_PERMISSIONS:
      return "Missing permissions to open the file."
  }
}
\end{minted}
\end{frame}

\begin{frame}[fragile]{Step 3 - custom error category}
\begin{minted}[highlightlines={2}]{c++}
std::string open_file_error_domain::message(int errc) const{
  if(errc < 0 or errc > 4) return "UNKNOWN ERROR";
  switch (static_cast<map_access_error>(errc)){
    case open_file_error::SUCCESS:
      return "Success.";
    case open_file_error::NO_SUCH_FILE_OR_DIRECTORY:
      return "File does not exist.";
    case open_file_error::FILE_IS_DIRECTORY:
      return "Cannot open directory.";
    case open_file_error::LACK_OF_RESOURCES:
      return "No available file descriptor";
    case open_file_error::FILE_BROKEN:
      return "File is corrupted."
    case open_file_error::NO_PERMISSIONS:
      return "Missing permissions to open the file."
  }
}
\end{minted}
\end{frame}

\begin{frame}[fragile]{Step 4 - factory function}

	\begin{minted}{c++}
namespace std{
  template <typename ErrorCode>
  error_code::error_code(typename std::enable_if<
                                  is_error_code_enum<
                                      ErrorCode>
                                  ::value, ErrorCode>
                         ::type errcode) noexcept 
             : error_code(make_error_code(errcode))
  {}
}
	\end{minted}
	
\end{frame}

\begin{frame}[fragile]{Step 4 - factory function}
	\begin{minted}{c++}
std::error_code make_error_code(open_file_error errc){
  return {static_cast<int>(errc), open_file_error_domain};
}
	\end{minted}
\end{frame}

\begin{frame}[fragile]{Step 5 - custom error condition}
	\begin{minted}{c++}
enum class library_error_condition : int {
  SUCCESS,
  WRONG_ARGUMENT,
  OS_ERROR,
  PERMISSIONS_ERROR
};
	\end{minted}
\end{frame}

\begin{frame}[fragile]{Step 5 - custom error condition}
	\begin{minted}{c++}

namespace std{
  template <> struct
  is_error_condition_enum<library_error_condition>
                             : std::true_type{};
}
	\end{minted}
\end{frame}

\begin{frame}[fragile]{Step 5 - custom error condition}
	\begin{minted}{c++}
struct library_error_domain : std::error_category{
  const char *name() const noexcept override;
  std::string message(int errc) const override;
  bool equivalent(const std::error_code &errc, int condition) 
                                      const noexcept override;
};
	\end{minted}
\end{frame}

\begin{frame}[fragile]{Step 5 - custom error condition}
	\begin{minted}{c++}
bool library_error_domain::equivalent(
          const std::error_code &errc, int condition) 
                                      const noexcept{
                                      
  switch (static_cast<library_error>(condition)){
    case library_error::SUCCESS:
      if(errc.value() == 0)
        return true;
    case library_error::WRONG_ARGUMENT:
      if(errc.category().name() == map_access_domain().name())
        return true;
        
    // other cases
  }
  return false;
}
	\end{minted}
\end{frame}

\begin{frame}[fragile]{Step 6 - Enjoy - real life example}
	\begin{minted}{c++}

std::error_code errcode;
auto route = calculate_route({}, {}, {}, errcode);

if(!errcode)
  return route;

std::cout << errcode.category().name() << " : " <<
             errcode.message() << std::endl;

if(errcode == calculate_route_error::MAP_ERROR)
  reinstall_map();
else if (errcode == calculate_route_error::COULD_NOT_FIND_PATH)
  inform_user_no_path_found();
else if (errcode == calculate_route_error::WRONG_ARGUMENTS)
  std::terminate();
	\end{minted}
\end{frame}

\begin{frame}[fragile]{Step 6 - Enjoy - real life example}
	\begin{minted}{c++}
route calculate_route(point a, point b, route_options options,
                      std::error_code& errc){
  auto map_database = database(errc);
  if (errc) return {};

  auto a_handle = map_database.find(a, errc);
  if(errc) return {};
  auto b_handle = map_database.find(b, errc);
  if(errc) return {};

  route result_route = find_path(a_handle, b_handle,
                                 options, errc);
  if(errc) return {};

  return result_route;
}
	\end{minted}
\end{frame}

	
\section{Error codes - summary}

\begin{frame}{Error codes summary}
	\begin{columns}[T]
		\begin{column}{0.48\linewidth}
			Pros 
			\vfill
			\begin{itemize}
				\item Performance
				\begin{itemize}
					\item speed
					\item small (occupied memory)
					\item speed predictability
					\item memory occupation predictability
					\item C compatibility
				\end{itemize}
			\end{itemize}
		\end{column}
		\begin{column}{0.48\linewidth}
			Cons
			\vfill
			\begin{itemize}
				\item business logic cluttering
				\item massive amount of boilerplate code
				\item template magic in case of std::error\_code
			\end{itemize}
		\end{column}
	\end{columns}
\end{frame}


\section{Exceptions to the rescue (?)}

\begin{frame}[fragile]{Brief look at the example}
	\begin{minted}{c++}
try{
  auto route = calculate_route(/*arguments*/);
} catch(map_error& err){
  // logic
} catch(path_not_found& err){
  // logic
} /* catch(std::invalid_argument){

} */
	\end{minted}
\end{frame}

\begin{frame}[fragile]{Brief look at the example}
	\begin{minted}{c++}
route calculate_route(point a, point b,
                      route_options options){
                      
  auto map_database = database();

  auto a_handle = map_database.find(a);
  auto b_handle = map_database.find(b);

  route result_route = find_path(a_handle, b_handle, options);

  return result_route;
}
	\end{minted}
\end{frame}

\begin{frame}[fragile]{Defining custom exception}
	\begin{minted}{c++}
class map_error : public std::runtime_error{};
	\end{minted}
\end{frame}

\begin{frame}[fragile]{Dark side of the exceptions}
	\begin{itemize}[<+- | alert@+>]
		\item Still translation of exceptions is needed
		\item For performance related reasons about 50\% of projects have disabled exceptions
	\end{itemize}
\end{frame}

\section{C++ - zero overhead rule}

\begin{frame}{What is zero overhead?}
	\begin{itemize}[<+- | alert@+>]
		\item language features {\color{amethyst}can} introduce overhead
		\item "you don't pay for what you don't use"
		\item if you use a feature it should be as afficient as handcoded version.
	\end{itemize}
\end{frame}

\begin{frame}{Exceptions not to the rescue}
		\centering
		\textcolor{red}{Exceptions break the zero overhead rule.}
		
		But why?
\end{frame}

\section{Exceptions - how do they work?}

\begin{frame}{Approaches towards implementation}
	Two major kinds of implementation:
	\begin{itemize}[<+- | alert@+>]
		\item additional data added to the frame stack
		\item additional data added to someplace on the heap
	\end{itemize}
\end{frame}
	
\begin{frame}{Implementations' consequences}
	\centering
	\begin{tabular}{p{3cm}|p{3cm}|p{3cm}}
		\multirow{2}{*}{implementation}& \multicolumn{2}{c}{performance} \\
		& without throwing & with throwing  \\ \hline \hline
		frame-based  & overhead & fast \\ \hline
		table-based & almost no overhead & slow \\ \hline
	\end{tabular}

	\vskip 1em

	\centering
	No matter the strategy binaries with exceptions enabled result in a big binary.
\end{frame}


\begin{frame}{Exceptions summary}
	\begin{columns}[T]
		\begin{column}{0.48\linewidth}
			Pros
			\begin{itemize}
				\item differentiated error and success paths
				\item automagical error propagation
				\item little/no boilerplate code
			\end{itemize}
		\end{column}
	\begin{column}{0.48\linewidth}
			Cons
			\begin{itemize}
				\item Performance
				\begin{itemize}
					\item slow
					\item not deterministic speed
					\item not deterministic storage occupation
					\item not compatible with C
					\item not usable in any safety standards (e.g. MISRA)
				\end{itemize}
			\end{itemize}
	\end{column}
	\end{columns}
\end{frame}

\section{Possible future of error handling.}

\begin{frame}{Perfect error handling mechanism}
	\centering
	\begin{tabular}{|c|c|c|}
		\hline
		feature & exceptions & std::error\_code \\ \hline \hline
		distinct error path & yes & no \\ \hline
		distinct success path & yes & no \\ \hline \hline
		unhandled error propagation & yes & no \\ \hline
		unhandled error is visible & no & yes \\ \hline
		uncluttered business logic & yes & no \\ \hline \hline
		RTTI required & yes & no \\ \hline
		deterministic space/time occupation & no & yes \\ \hline
		time cost == return & no & yes \\ \hline \hline
		C compatibility & no & no \\ \hline
	\end{tabular}
\end{frame}

\begin{frame}{Key idea for improvement}
	\begin{block}{Key ideas}
		\begin{itemize}[<+- | alert@+>]
			\item Let's use the return channel to return the std::error
			\item Let the compiler generate boilerplate code for error propagation
		\end{itemize}
	\end{block}
	Let's call those \emph{\color{amethyst}static exceptions}
\end{frame}
	
\begin{frame}{What is std::error}
	\begin{itemize}
		\item size of error\_code is explicitly defined
		\item has same properties as std::error\_code
		\item implements trivially relocatable semantics
	\end{itemize}
\end{frame}
	
\begin{frame}[fragile]{How to use return channel for std::error}
	\begin{columns}[T]
		\begin{column}{0.48\linewidth}
			We can do that manually using variant:
			\begin{minted}{c++}
using Result = 
      variant<T, std::error>;			
Result foo();
			\end{minted}
		\end{column}
		\begin{column}{0.48\linewidth}
			But this can be nicer with syntax sugar:
			\begin{minted}{c++}
T foo() throws;
			\end{minted}
		\end{column}
	\end{columns}
\end{frame}

\begin{frame}[fragile]{Static exceptions example}
	
	\begin{minted}{c++}
string f() throws {
  if (flip_a_coin()) throw arithmetic_error::something;
  return "xyzzy"s + "plover"; 
}

string g() throws { return f() + "plugh"; } 

int main() {
  try {
    auto result = g();
    cout << "success, result is: " << result;
  } catch(error err) {
    cout << "failed, error is: " << err.error();
  }
}
	\end{minted}
\end{frame}

\begin{frame}[fragile]{Compiler's support in handling errors}
	\begin{minted}{c++}
int f1() throws;	
int f2() throws;

int main(){
// return f1() + f2(); // error
try return f1() + f2();// ok, covers both 
return try f1()+ f2();
}
	\end{minted}
\end{frame}

\begin{frame}{Cool! But what about C compatibility}
	It is possible, that C will be ABI compatible with static exceptions!
	
	This implies:
	
	\begin{itemize}
		\item Exceptions could be thrown from C++, passed through C and catched again in C++
		\item We could handle C++ exceptions in C
		\item We could handle C exceptions in C++
	\end{itemize}
\end{frame}

\begin{frame}[fragile]{Short story about C language}
	\begin{minted}{c++}
_Either(int, std_error) somefunc(int a){ fixme
  return a > 5 ? _Expected(a) : _Unexpected(a);
}

// ...

_Either(int, std_error) ret = somefunc(a);
if(ret)
  printf("%d\n", ret.expected);
else
  printf("%f\n", ret.unexpected);
	\end{minted}
\end{frame}

\begin{frame}{Static exceptions summary}
\centering
	\begin{tabular}{|c|c|}
		\hline
		feature & static exceptions \\ \hline \hline
		distinct error path & yes \\ \hline
		distinct success path & yes \\ \hline \hline
		unhandled error propagation & yes \\ \hline
		unhandled error is visible & yes \\ \hline
		uncluttered business logic & yes \\ \hline \hline
		RTTI required & no \\ \hline
		deterministic space/time occupation & yes \\ \hline
		time cost == return & yes \\ \hline \hline
		C compatibility & maybe \\ \hline
	\end{tabular}
\end{frame}

\begin{frame}{Possible issue}
	We will end up having 3 ways to handle error codes:
	\begin{itemize}
		\item dynamic exceptions
		\item static exceptions
		\item old style error codes
	\end{itemize}

	\vfill

	\texttt{std::error} type can be treated as a next-gen \texttt{std::error\_code} type.
\end{frame}

\begin{frame}{Bibliography}
	This presentation wouldn't be possible without:
	
	\begin{itemize}
		\item Herb Sutter - author of the proposal (code examples, exception features taken from his proposal) -  \href{http://www.open-std.org/jtc1/sc22/wg21/docs/papers/2018/p0709r1.pdf}{p0709r1}
		
		\item Andrzej Krzemiński for his blog about error codes and error conditions -  \href{https://akrzemi1.wordpress.com/2017/07/12/your-own-error-code/}{Your own error code}
	\end{itemize}
\end{frame}

\begin{frame}{Thank you}
	\centering
	Thank you for your attention!
	\vfill
	Questions?
\end{frame}



\begin{frame}{The END}
\centering
	Thanks for attention!
\end{frame}
	
\end{document}
